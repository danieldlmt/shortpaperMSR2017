
\documentclass[10pt, conference]{IEEEtran}


% *** CITATION PACKAGES ***
%
%\usepackage{natbib}
%\usepackage{cite}
% cite.sty was written by Donald Arseneau
% V1.6 and later of IEEEtran pre-defines the format of the cite.sty package
%$ \cite{} output to follow that of IEEE. Loading the cite package will
% result in citation numbers being automatically sorted and properly
% "compressed/ranged". e.g., [1], [9], [2], [7], [5], [6] without using
% cite.sty will become [1], [2], [5]--[7], [9] using cite.sty. cite.sty's
% \cite will automatically add leading space, if needed. Use cite.sty's
% noadjust option (cite.sty V3.8 and later) if you want to turn this off.
% cite.sty is already installed on most LaTeX systems. Be sure and use
% version 4.0 (2003-05-27) and later if using hyperref.sty. cite.sty does
% not currently provide for hyperlinked citations.
% The latest version can be obtained at:
% http://www.ctan.org/tex-archive/macros/latex/contrib/cite/
% The documentation is contained in the cite.sty file itself.




% *** GRAPHICS RELATED PACKAGES ***
%
\ifCLASSINFOpdf
   \usepackage[pdftex]{graphicx}
  % declare the path(s) where your graphic files are
   \graphicspath{{../pdf/}{../jpeg/}}
  % and their extensions so you won't have to specify these with
  % every instance of \includegraphics
   \DeclareGraphicsExtensions{.pdf,.jpeg,.png}
\else
  % or other class option (dvipsone, dvipdf, if not using dvips). graphicx
  % will default to the driver specified in the system graphics.cfg if no
  % driver is specified.
  % \usepackage[dvips]{graphicx}
  % declare the path(s) where your graphic files are
  % \graphicspath{{../eps/}}
  % and their extensions so you won't have to specify these with
  % every instance of \includegraphics
  % \DeclareGraphicsExtensions{.eps}
\fi
% graphicx was written by David Carlisle and Sebastian Rahtz. It is
% required if you want graphics, photos, etc. graphicx.sty is already
% installed on most LaTeX systems. The latest version and documentation can
% be obtained at: 
% http://www.ctan.org/tex-archive/macros/latex/required/graphics/
% Another good source of documentation is "Using Imported Graphics in
% LaTeX2e" by Keith Reckdahl which can be found as epslatex.ps or
% epslatex.pdf at: http://www.ctan.org/tex-archive/info/
%
% latex, and pdflatex in dvi mode, support graphics in encapsulated
% postscript (.eps) format. pdflatex in pdf mode supports graphics
% in .pdf, .jpeg, .png and .mps (metapost) formats. Users should ensure
% that all non-photo figures use a vector format (.eps, .pdf, .mps) and
% not a bitmapped formats (.jpeg, .png). IEEE frowns on bitmapped formats
% which can result in "jaggedy"/blurry rendering of lines and letters as
% well as large increases in file sizes.
%
% You can find documentation about the pdfTeX application at:
% http://www.tug.org/applications/pdftex


\usepackage[colorinlistoftodos]{todonotes}



% correct bad hyphenation here
\hyphenation{op-tical net-works semi-conduc-tor}


\begin{document}
%
% paper title
% can use linebreaks \\ within to get better formatting as desired
\title{Characterizing the development of a cross-platform library: a case study of Allegro}
% by mining commit logs
% Mining and Characterizing Cross-Platform libraries


% author names and affiliations
% use a multiple column layout for up to two different
% affiliations

\author{\IEEEauthorblockN{Authors Name/s per 1st Affiliation (Author)}
\IEEEauthorblockA{line 1 (of Affiliation): dept. name of organization\\
line 2: name of organization, acronyms acceptable\\
line 3: City, Country\\
line 4: Email: name@xyz.com}
\and
\IEEEauthorblockN{Authors Name/s per 2nd Affiliation (Author)}
\IEEEauthorblockA{line 1 (of Affiliation): dept. name of organization\\
line 2: name of organization, acronyms acceptable\\
line 3: City, Country\\
line 4: Email: name@xyz.com}
}

% conference papers do not typically use \thanks and this command
% is locked out in conference mode. If really needed, such as for
% the acknowledgment of grants, issue a \IEEEoverridecommandlockouts
% after \documentclass

% for over three affiliations, or if they all won't fit within the width
% of the page, use this alternative format:
% 
%\author{\IEEEauthorblockN{Michael Shell\IEEEauthorrefmark{1},
%Homer Simpson\IEEEauthorrefmark{2},
%James Kirk\IEEEauthorrefmark{3}, 
%Montgomery Scott\IEEEauthorrefmark{3} and
%Eldon Tyrell\IEEEauthorrefmark{4}}
%\IEEEauthorblockA{\IEEEauthorrefmark{1}School of Electrical and Computer Engineering\\
%Georgia Institute of Technology,
%Atlanta, Georgia 30332--0250\\ Email: see http://www.michaelshell.org/contact.html}
%\IEEEauthorblockA{\IEEEauthorrefmark{2}Twentieth Century Fox, Springfield, USA\\
%Email: homer@thesimpsons.com}
%\IEEEauthorblockA{\IEEEauthorrefmark{3}Starfleet Academy, San Francisco, California 96678-2391\\
%Telephone: (800) 555--1212, Fax: (888) 555--1212}
%\IEEEauthorblockA{\IEEEauthorrefmark{4}Tyrell Inc., 123 Replicant Street, Los Angeles, California 90210--4321}}




% use for special paper notices
%\IEEEspecialpapernotice{(Invited Paper)}




% make the title area
\maketitle

%%%%%%%%%%%%%%%%%%%%%%%%%%%%%%%%%%%%%%%%
%%%%%%%%%%%%%%%%%%%%%%%%%%%%%%%%%%%%%%%%%
%%%%%%%%%%%%%%%%%%%%%%%%%%%%%%%%%%%%%%%%%
\begin{abstract}
As new operating systems and hardware devices emerge and become popular, developers demand libraries to help them build cross-platform applications. Nevertheless, there is a lack of knowledge about the development process of such libraries. With the goal of characterizing the development of cross-platform libraries, we analyzed the commit log of Allegro, a successful cross-platform multimedia library written in C. We found that 70\% of the developers contribute to two or more platforms, although only 20\% write code for both desktop and mobile devices. Moreover, we identified that core developers contribute to significantly more platforms and device types than peripheral developers. As future work, we plan to replicate this study in order to validate our conclusions with other cross-platforms libraries.

\end{abstract}

\begin{IEEEkeywords}
cross-platform development; software libraries; version control system.

\end{IEEEkeywords}


% For peer review papers, you can put extra information on the cover
% page as needed:
% \ifCLASSOPTIONpeerreview
% \begin{center} \bfseries EDICS Category: 3-BBND \end{center}
% \fi
%
% For peerreview papers, this IEEEtran command inserts a page break and
% creates the second title. It will be ignored for other modes.
\IEEEpeerreviewmaketitle

% Challenges of running / maintaining a cross-platform library project -- specialization of developers
  % We conjecture that successful projects are led by developers who master multiple platforms
  % Some platforms have more isolated communities, making it difficult to find developers for those platforms that also develop for other platforms
  % If you know platform X, you're likely to know platform Y, making it easier to find people who know both X and Y
% platform is more independent => the code is very different from other platforms


%%%%%%%%%%%%%%%%%%%%%%%%%%%%%%%%%%%%%%
%%%%%%%%%%%%%%%%%%%%%%%%%%%%%%%%%%%%%%
%%%%%%%%%%%%%%%%%%%%%%%%%%%%%%%%%%%%%%
\section{Introduction}
% no \IEEEPARstart
\todo[inline]{Continuo achando que essa primeira frase esta estranha. Veja os comentarios em https://goo.gl/PD2Mei}
Cross-platform programming has been used to develop several libraries, toolkits, frameworks that allow developers writing multiplatform applications even though the developer is not familiarized with technical aspects about the target platform. Each platform can have a unique approach to implement a functionality or communicate with a hardware device. Hence to build a multiplatform application it’s necessary to write platform specific code to handle with the different hardware architecture and implementations of file system, graphic user interface (GUI), audio system, security [ ]. 

Compiled languages require compiling for each platform it supports. Libraries such as Simple DirectMedia Layer (SDL) and Allegro have implementations for multiple platforms and allow developers to code for the library used instead of the target platforms.     

In this paper we are interested to know some characteristics about cross-platform library projects. Thus we selected two cross-platform libraries to work on and we use the  information available in the repository to answer the following research questions:

\todo[inline]{Pode chamar de RQ1, RQ2 e RQ3, para facilitar referenciar as questoes em outras partes do texto}

\begin{itemize}
% \item  Which platform is more independent in terms of being modified alone during maintenance tasks? \todo[inline]{Which platforms tend to change when there is a change in platform-independente code?}
\item Do core developers tend to contribute with more platforms than peripheral developers?
\item Is there a clear separation between mobile and desktop developers?
\item  Which platforms are more related together considering the set of platforms a developer work with? 
\end{itemize}

\todo[inline]{Este paragrafo deveria estar no inicio.}
Cross-platform libraries, toolkits and frameworks are in demand because of the increasing number of hardware devices and Operating Systems (OS). Companies want to make their products available for as many platforms as possible but developing a multiplatform application can be costly. Cross-platform programming is not a very explored area although there is a lot of challenges involving multiplatform applications.   

The remainder of this paper is organized as follows: Section X…. 

\section{Background}

To achieve the goal of building a cross-platform application in the scenario of many device types and platforms, developers should consider three fundamental requites \cite{Levin2014}: an application should be consistent, continuous and complementary. Consistent design means that the core functionalities and structure are kept across platforms. Continuous design is when a user can start an activity in one platform and finish it in some other platform. Complementary design is that devices complement one another with a relevant functionality that only a specific device can do.     

\todo[inline]{Nao entendi de que forma cross-platform aumenta a qualidade do codigo. Essa afirmacao veio de algum trabalho relacionado?}

Building cross-platform raises the overall quality of code and the range of users who can take advantage of the software. A software engineering of Backblaze reported 10 rules for writing cross-platform code in C and C++ \cite{backblaze2008}:

\todo[inline]{Acho que nao deve colocar essa lista aqui, porque (1) voce nao vai explicar tudo, e sem explicacao o leitor nao vai entender o que cada coisa significa e (2) nem tudo eh relevante para seu trabalho.}

\begin{enumerate}
\item Simultaneously development - Think about cross-platform from the very beginning of a project.  
\item XML for GUI - Factor out the GUI into non reusable code and then develop a cross-platform library for the underlying logic.
\item Use standard C types, not platform-specific types.
\item Use only built in \#ifdef compiler flags, do not invent your own.
\item Develop a simple set of reusable, cross-platform base libraries to hide platform-specific code
\item Use Unicode, specifically UTF-8, for all APIs.
\item Don’t use 3rd party application frameworks or runtime environments. 
\item Build the raw source directly on all platforms.
\item Require all programmers to compile on all platforms.
\item Fire the lazy, incompetent, or bad attitude programmers who can’t follow these rules.
\end{enumerate}

Avaya's projects use fork to outsource development team. As developers generally support a few number of platforms, Avaya's development team  was divided in pure desktop and Windows team and the purely mobile teams. Consequently, it hinders fixing bugs  in all supported platforms and create redundant development effort in some cases \cite{Duc2014}.  

\todo[inline]{Podia aproveitar e fazer um gancho, dizendo que neste trabalho voce vai caracterizar um projeto open source bem-sucedido quando aa atribuicao de plataformas a desenvolvedores.}


%%%%%%%%%%%%%%%%%%%%%%%%%%%%%%%%%%%%%%
%%%%%%%%%%%%%%%%%%%%%%%%%%%%%%%%%%%%%%
%%%%%%%%%%%%%%%%%%%%%%%%%%%%%%%%%%%%%%
\section{Methodology}

\todo[inline]{Use "Table I shows..." (T maiusculo)}

In order that we could establish relations among platforms and know better the developer team profile of cross-platform library projects, we selected Allegro library to conduct the study. The table \ref{allegrogeneral} shows a summary of Allegro’s general characteristics.  

\begin{table}[h]
%% increase table row spacing, adjust to taste
\renewcommand{\arraystretch}{1.3}
\caption{Allegro’s general characteristics}
\label{allegrogeneral}
\centering

\begin{tabular}{|c|c|}
\hline
Type & Multimedia and Games SDK \\
\hline
First release & 1990 \\
\hline
Language & C \\
\hline
Licence & zlib \\
\hline
Platforms & Windows, Linux, Mac OS X, iPhone, Android \\
\hline
\end{tabular}
\end{table}

We extracted the commits related to modifications from Allegro's repository. The table \ref{allegroinfo} shows the release selected, some size metrics, the analysis period  and the quantity of commits extracted from Allegro's repository.  

\begin{table}[h]
\renewcommand{\arraystretch}{1.3}
\caption{Allegro’s development informationm}
\label{allegroinfo}
\centering
\begin{tabular}{|c|c|c|}
\hline
Release & 5.2.1.1\\
\hline
LOC & 79.004 \\
\hline
Modules & 144\\
\hline
Packages & 13\\
\hline
Functions & 2.358\\
\hline
Commits & 1.143  \\
\hline
Period & 01/12/2011 - 01/12/2016   \\
\hline
\end{tabular}
\end{table}

\todo[inline]{Table 2, adicionar: number of developers}

Allegro's first release supported Windows, Mac OSX, and Linux. Then in August 2009 they started supporting Iphone and in December 2011 they started supporting Android. As our intention was to study all five platforms, we extracted the commits from 1st December 2011 until 1st December 2016, totalizing 1143 commits.  

For the data mining and quantitative analysis we used the R, programming language and software environment for statistical computing and graphics. The study was designed as a three-part method, as illustrated in figure \ref{metodologia}: 

\begin{itemize}
\item Commit log extraction: Allegro uses Git, an open version control repository (VCS), and its commit log was freely obtained from the repository.
\item  Data mining: The R was used to extract information about the platforms and developer team from Allegro's commit log. 
\item Quantitative analysis: The analysis were also made in the R. For each research question a different analysis was required, as will be further explained in sections \ref{met_analysis1}, \ref{met_analysis2} and \ref{met_analysis3}.
\end{itemize}

\begin{figure}[h]
\centering
\textbf{}\includegraphics[width=2.5in]{metodologia}
\caption{Study fluxogram}
\label{metodologia}
\end{figure}

To identify the platforms a developer works with, we listed all platform-specific code the developer has touched in the commits and it was considered that a developer supports a certain platform if the developer has made at least one modification in the platform-specific code related to this platform.

\todo[inline]{Faltou explicar como voce sabe qual plataforma o commit esta tocando. Eh pela estrutura de diretorios.}

Despite Allegro supports officially five platforms, as shown in table \ref{allegrogeneral}, we considered Unix as a sixth platform as Linux is based on Unix and we also discovered that many other packages are logically coupled with Unix package. Figure \ref{diagrama} shows a UML diagram of logical coupling among packages of Allegro. 

\begin{figure}[h]
\centering
\textbf{}\includegraphics[width=2.5in]{diagrama}
\caption{UML diagram of logical coupling among packages}
\label{diagrama}
\end{figure} 


%%%%%%%%%%%%%%%%%%%%%%%%%%%%%%%
\subsubsection{Analysis: Do core developers tend to contribute with more platforms than peripheral developers?}
\label{met_analysis1}

In order to know if core developers tend to contribute with more platforms than peripheral developers, we divided the developers in core and peripheral and counted the number of platforms each developer works with. 

The division was made slicing the developer team by classifying the top 20\% in the core group and the remaining in the peripheral group \cite{Robles2009}. Using this approach, the core group was responsible for 88,4\% of the total number of commits.

 




%%%%%%%%%%%%%%%%%%%%%%%%%%%%%%%%%
\subsubsection{Analysis: Is there a clear separation between mobile and desktop developers?}
\label{met_analysis2}

\todo[inline]{Essa explicacao abaixo nao ficou clara. Para cada desenvolvedor, voce contou os commits para plataformas mobile e para plataformas desktop. Do jeito que esta explicado, parece que voce pegou todos os commits e dividiu em mobile e desktop, sem considerar a divisao por desenvolvedor.}


We divided the commits according to the device type: mobile and desktop. Therefore all the analysis were made considering Android and Iphone in mobile group and Linux, Unix, Windows and Mac OSX in desktop group. 



%%%%%%%%%%%%%%%%%%%%%%%%%%%%%%%%%%%%%%
\subsubsection{Analysis: Which platforms are more related together considering the set of platforms a developer work with? }
\label{met_analysis3}

To know which platforms are more related together considering the set of platforms a developer work with, we organized the data by developer and listed all platforms the developer works with and then we applied the Apriori algorithm for mining frequent set of platforms.



The Apriori algorithm is a widely used association rule algorithm and its principle is that "if an itemset is frequent, then all of its subsets must also be frequent".
\todo[inline]{Esse principio tem a ver com o funcionamento do algoritmo, que nao eh tao relevante para este paper.}
Using Apriori we intended to verify how many times Windows and Mac OSX packages were modified in the same commit, for example. 
\todo[inline]{O Apriori faz um pouco mais que isso, pois ele enxerga relacionamentos assimetricos.}

Two important parameters of Apriori are support and confidence. Support represents how often a rule occurs in a given dataset, while confidence determines how frequently two subset appear together in a transaction. We used a trial and error method to find the most suitable values for support and confidence and we determined support=0,3 and confidence=0,8.     

\todo[inline]{Quem nao conhece o Apriori nao vai entender. Precisa mostrar exemplos de rules.}

\todo[inline]{``Trial and error method'' nao soa muito cientifico...}

%%%%%%%%%%%%%%%%%%%%%%%%%%%%%%%%%%%%%%
%%%%%%%%%%%%%%%%%%%%%%%%%%%%%%%%%%%%%%
%%%%%%%%%%%%%%%%%%%%%%%%%%%%%%%%%%%%%%
\section{Results}
In this section we present the results obtained in the case study about characterization of the development of cross-platform libraries. 

As a preprocessing step, with the aim of verifying the modularity and communication among packages of Allegro, we built a UML diagram of logical coupling among packages, shown in figure \ref{diagrama}, and we counted how many packages were modified in each commit. The table  \ref{packagegeneral} presents amount of packages modified in each commit. 

\begin{table}[h]
\renewcommand{\arraystretch}{1.3}
\caption{Number of packages modified in a commit}
\label{packagegeneral}
\centering
\begin{tabular}{|c|c|c|}
\hline
 Package & Commit & \%\\
\hline
1&		964&	84,3\\
2&		98&		8,6\\
3&		46& 	4,9\\
4&		14&		1,2\\
5&		9&		0,8\\
6&		7&		0,6\\
7&		3&		0,3\\
8&		2&		0,2\\

\hline
\end{tabular}
\end{table}

We found that 84,3\% of commits touches only one package, it suggests that Allegro is a well modularized system. 

The diagram of logical coupling shows some unexpected relationship among packages:

\begin{itemize}
\item Mac OSX package doesn't communicate with any other package.
\item  Android package is accessed by Allegro package.
\end{itemize}

\todo[inline]{A meu ver essa analise de pacotes esta deslocada. Voce esta respondendo a uma pergunta que voce nao fez na introducao e metodologia. Alem disso, por que reportar resultados de pacotes se voce esta estudando plataformas? Voce pode reportar essa analise com plataformas, mas nesse caso deveria colocar isso como questao de pesquisa.}

%%%%%%%%%%%%%%%%%%%%%%%%%%%%%%%%
\subsubsection{Results: Do core developers tend to contribute with more platforms than peripheral developers?}

The figure \ref{DeveloperSpecialization} shows how many platforms a developer supports, in general. In figure \ref{DeveloperSpecialization} we can see that see that about 50\% support more than one platform. 

\todo[inline]{Na figura 3 nao da pra ver essa informacao sobre 50\% diretamente, a nao ser que o leitor some todas as contagens do histograma.}

\begin{figure}[h]
\centering
\includegraphics[width=2.5in]{DeveloperSpecialization}
\caption{Developer specialization}
\label{DeveloperSpecialization}
\end{figure}

Dividing the developers in core and peripheral team, we set the core team with 4 developers and the peripheral team with 17 developers. The core team was responsible for 97,4\% of the total number of commits. This information can be seen in table \ref{devgrouping}.  

\todo[inline]{La em cima voce falou em 88,4\%.}
\todo[inline]{Use . como separador decimal.}
\todo[inline]{Nao precisa de uma tabela com a mesma informacao deste paragrafo, sendo que essa informacao nao eh o seu resultado, eh apenas uma contextualizacao.}

\begin{table}[h]
\renewcommand{\arraystretch}{1.3}
\caption{Developer grouping}
\label{devgrouping}
\centering
\begin{tabular}{|c|c|c|}
\hline
  & Team size   & Commits (\%) \\
\hline
Core & 4& 97,4 \\
\hline
Peripheral & 17 & 2,6\\
\hline
\end{tabular}
\end{table} 

Our investigation shows that all core developers work with all six platforms and that peripheral developers tend to be specialized in a single platform, as 47,1\% of them work with one platform only. We also found that 11,47\% of peripheral developers support just the platform-independent code. The table \ref{plat} presents all information about the number of platform core and peripheral developer support.    

\todo[inline]{Eh a primeira vez que voce fala sobre platform-independent code; isso nao foi explicado antes.}

\begin{table}[h]
\renewcommand{\arraystretch}{1.3}
\caption{Number of platform core and peripheral developer support}
\label{plat}
\centering
\begin{tabular}{|c|c|c|}
\hline
Number of platform & Core (\%) & Peripheral(\%)\\
\hline
0 & 	- & 	11,47 \\
\hline
1 & 	- & 	47,1 \\
\hline
2 & 	-& 		17,6\\
\hline
3 &		- &		5,9\\
\hline
4 & 	- & 	11,8\\
\hline
5 & 	- & 	5,9  \\
\hline
6 & 	100 & 	-  \\
\hline
\end{tabular}
\end{table} 






%%%%%%%%%%%%%%%%%%%%%%%%%%%%%%%%%
\subsubsection{Results: Is there a clear separation between mobile and desktop developers?}

The table \ref{devicetype} shows the device type specialization of core and peripheral developers. The results show that there is no separation between mobile and desktop developers in team core team, as all of them support both device types. But in peripheral team, we can see a small segregation although 41,18\% of the team support both platforms.       

\todo[inline]{O caption da Tabela VI deveria permitir ao leitor entender a tabela sem ler o texto fora da tabela. Em particular, deveria explicar o que sao as porcentagens.}

\begin{table}[h]
\renewcommand{\arraystretch}{1.3}
\caption{Device type specialization}
\label{devicetype}
\centering
\begin{tabular}{|c|c|c|}
\hline
 Device type & Core (\%) & Peripheral (\%) \\
\hline
None & 			- & 		11,8   \\
\hline
Desktop & 		- & 		29,4  \\
\hline
Mobile & 		- & 		17,4 \\
\hline
Desktop and Mobile & 100 & 41,18  \\
\hline

\end{tabular}
\end{table} 






%%%%%%%%%%%%%%%%%%%%%%%%%%%%%%%%
\subsubsection{Results: Which platforms are more related together considering the set of platforms a developer work with? }

We verified how many packages are touched in a commit for each platform, the results are shown in figure \ref{packagesbyplatform}. The figure \ref{packagesbyplatform} can be interpreted as: when a platform package is touched in a maintenance task, how many packages are also modified? Except Unix that have the median equals to two, all other five platforms have the minimum and median equals to 1, with special attention to Mac OSX that also have maximum equal to 1.  

\todo[inline]{1 nao eh o maximo do Mac OS X, eh o seu terceiro quartil.}
\todo[inline]{Mais uma vez aparece o numero de pacotes em uma analise, resquicio de analises anteriores, mas que nao cabe na proposta deste artigo. Se fosse plataforma vs plataforma, melhor.}

\begin{figure}[h]
\centering
\includegraphics[width=3.0
in]{packagesbyplatform}
\caption{Number of packages touched in a commit for each platform}
\label{packagesbyplatform}
\end{figure}

The Apriori association rule results can be seen in table \ref{assocrule}. The most frequent rule found was the rule 1, that says that when a developer works with Iphone, the developer also works with Windows with a confidence of 0,89. The rule 1 was true for 8 out of 21 transactions.
The rule 2, 3 and 4 was found in 7 out of 21  transactions with a confidence of 1,00. The rule 5 was found in 7 out of 21 transactions with a confidence of 0,88. 

\todo[inline]{Transactions sao um termo de association rule? Melhor falar de commits.}

\todo[inline]{Quando voce diz out of 21, voce nao quer dizer que o total sao 21 commits, certo?}

\todo[inline]{Vale a pena comentar que essa associacao entre iPhone e Windows eh contra-intuitiva.}


\begin{table}[h]
\renewcommand{\arraystretch}{1.3}
\caption{Apriori association rule}
\label{assocrule}
\centering
\begin{tabular}{|c|c c c|c|c|}
\hline
 Rule	&LHS & &RHS	& Support & Confidence \\
\hline
1&	\{Iphone\} 			&$\Rightarrow$	& 	\{Windows\} 	& 	0,3809524	&  	0,89	\\
\hline
2&	\{Linux\} 			& $\Rightarrow$ & 	\{android\} 	& 	0,3333333	&  	1,00 	\\
\hline
3&	\{Iphone,Macosx\}	&$\Rightarrow$	& 	\{Windows\} 	& 	0,3333333	& 	1,00	\\
\hline
4&	\{Windows,Macosx\} &$\Rightarrow$&  	\{Iphone\} 	& 	0,3333333	& 	1,00	\\
\hline
5&	\{Windows,Iphone\} &$\Rightarrow$& 		\{Macosx\} 	& 	0,3333333	&  	0,88	\\
\hline
\end{tabular}
\end{table} 





%%%%%%%%%%%%%%%%%%%%%%%%%%%%%%%%%%%%%%
%%%%%%%%%%%%%%%%%%%%%%%%%%%%%%%%%%%%%%
%%%%%%%%%%%%%%%%%%%%%%%%%%%%%%%%%%%%%%
\section{Conclusion}
The conclusion goes here. this is more of the conclusion

\todo[inline]{Acho que vale a pena mencionar brevemente as ameacas aa validade aqui na conclusao.}


%\cite{Robles2009} % Core peripheral developer
%\cite{backblaze2008} % ten rules
%\cite{Levin2014} %Book
%\cite{Duc2014} ¨% Forking and Coordination

\cite{bishop2006} % software that lasts
\cite{Wojtczyk2008} % A Cross Platform Development Workflow for C/C++ Applications
\cite{Fahy2012} %Using open source libraries in cross platform games development
\cite{Nebeling2013} % Informing the Design of New Mobile Development Methods and Tools
\cite{Sorensen2014} % The 4C Framework: Principles of Interaction in Digital Ecosystems
\cite{Dong2016} %Understanding the Challenges of Designing and Developing Multi-Device Experiences

% conference papers do not normally have an appendix


% use section* for acknowledgement
\section*{Acknowledgment}


The authors would like to thank...
more thanks here


% trigger a \newpage just before the given reference
% number - used to balance the columns on the last page
% adjust value as needed - may need to be readjusted if
% the document is modified later
%\IEEEtriggeratref{8}
% The "triggered" command can be changed if desired:
%\IEEEtriggercmd{\enlargethispage{-5in}}

% references section

% can use a bibliography generated by BibTeX as a .bbl file
% BibTeX documentation can be easily obtained at:
% http://www.ctan.org/tex-archive/biblio/bibtex/contrib/doc/
% The IEEEtran BibTeX style support page is at:
% http://www.michaelshell.org/tex/ieeetran/bibtex/
\bibliographystyle{IEEEtran}
% argument is your BibTeX string definitions and bibliography database(s)
\bibliography{bib}
%
% <OR> manually copy in the resultant .bbl file
% set second argument of \begin to the number of references
% (used to reserve space for the reference number labels box)
%\begin{thebibliography}{1}



%\bibitem{IEEEhowto:kopka}
%H.~Kopka and P.~W. Daly, \emph{A Guide to \LaTeX}, 3rd~ed.\hskip 1em plus
 % 0.5em minus 0.4em\relax Harlow, England: Addison-Wesley, 1999.

%\end{thebibliography}




% that's all folks
\end{document}


